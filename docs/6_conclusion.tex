\section*{\large Заключение}
\addcontentsline{toc}{chapter}{Заключение}
    \par Цель курсовой работы достигнута -- было реализовано программное обеспечение для визуализации модели планетарной системы. Пользователь может динамически просматривать сцену: масштабировать, запускать и ставить на паузу движение модели планетарной системы.
	\par В ходе выполнения работы:
	\begin{enumerate}
		\item были проанализированы существующие алгоритмы компьютерной графики — были рассмотрены алгоритмы удаления невидимых ребер и поверхностей, алгоритмы закраски полигонов и методы моделирования освещения. Существующие алгоритмы были адаптированы для решения поставленной задачи;
		\item были реализованы выбранные алгоритмы удаления невидимых ребер и поверхностей, алгоритмы закраски полигонов и методы моделирования освещения;
		\item был проведен эксперимент, определяющий зависимости процессорного времени генерации изображения от количества точек, аппроксимирующих поверхности объектов. Было определено, что эта зависимость стремятся к квадратичной.
	\end{enumerate}
	\par В дальнейшем этот продукт можно улучшить -- использовать реалистичные текстуры для изображения поверхностей космических объектов, добавить звезды, кольца планет и астероиды.
\newpage
